% Section 4.6 - Kiến trúc vật lý
\section{Kiến trúc vật lý}
\label{sec:physical-architecture}

Kiến trúc vật lý mô tả cách các thành phần phần mềm được ánh xạ lên các nút vật lý trong môi trường vận hành thực tế. Phần này trình bày sơ đồ triển khai, các thành phần hạ tầng và luồng giao tiếp giữa chúng.

\subsection{Sơ đồ triển khai}
\label{subsec:deployment-diagram}

Hệ thống được triển khai theo mô hình kiến trúc client-server, trong đó thành phần backend được xây dựng dưới dạng ứng dụng đơn khối (\textit{monolithic application}) và tích hợp với các dịch vụ bên ngoài. Mô hình này phù hợp với quy mô hiện tại của hệ thống, đồng thời đơn giản hóa quá trình phát triển, kiểm thử và triển khai.

Hình~\ref{fig:deployment-diagram1} thể hiện sơ đồ triển khai tổng quan của hệ thống với năm thành phần chính. Thành phần Client là ứng dụng di động Flutter đa nền tảng, đóng vai trò giao diện tương tác với người dùng cuối. API Server được xây dựng trên nền tảng Node.js với framework Express, chịu trách nhiệm xử lý toàn bộ logic nghiệp vụ theo mô hình Clean Architecture đã trình bày ở các phần trước. PostgreSQL được lựa chọn làm hệ quản trị cơ sở dữ liệu quan hệ để lưu trữ dữ liệu nghiệp vụ. Recommend Service là một dịch vụ độc lập được xây dựng bằng Python, chạy theo lịch định kỳ để tính toán vector nhúng và sinh gợi ý việc làm. Dịch vụ này được kích hoạt bởi AWS EventBridge Scheduler và chạy trên AWS Lambda, kết nối trực tiếp với cơ sở dữ liệu để đọc thông tin CV và công việc, sau đó lưu kết quả gợi ý vào database. Cuối cùng, Firebase Storage đảm nhận vai trò lưu trữ các tệp tin đa phương tiện như ảnh đại diện và CV định dạng PDF.

\begin{figure}[H]
    \centering
    \includegraphics[width=0.95\linewidth]{Figures/Chương 4 Thiết kế/41/Deployment Diagram.png}
    \caption{Sơ đồ triển khai hệ thống}
    \label{fig:deployment-diagram1}
\end{figure}

\subsection{Luồng giao tiếp giữa các thành phần}
\label{subsec:communication-flow}

Các thành phần trong hệ thống giao tiếp với nhau thông qua các giao thức chuẩn hóa. Phần này mô tả chi tiết luồng giao tiếp giữa từng cặp thành phần.

Giao tiếp giữa Client và API Server được thực hiện thông qua giao thức HTTPS với kiến trúc RESTful API. Mọi yêu cầu từ Client đều được gửi đến API Server dưới dạng HTTP request, trong đó các endpoint công khai như đăng nhập và đăng ký không yêu cầu xác thực, còn các endpoint được bảo vệ yêu cầu JWT token hợp lệ trong header Authorization. Cơ chế này đảm bảo tính bảo mật trong quá trình trao đổi dữ liệu giữa ứng dụng di động và máy chủ.

API Server tương tác với cơ sở dữ liệu PostgreSQL thông qua Prisma ORM. Prisma cung cấp lớp trừu tượng hóa an toàn kiểu, cho phép định nghĩa mô hình dữ liệu trong mã nguồn và tự động đồng bộ với lược đồ cơ sở dữ liệu. Việc sử dụng ORM giúp giảm thiểu lỗi truy vấn SQL và tăng năng suất phát triển.

Recommend Service hoạt động độc lập và được kích hoạt theo lịch định kỳ thông qua AWS EventBridge Scheduler. Khi được trigger, AWS Lambda sẽ khởi chạy dịch vụ để thực hiện quy trình sinh gợi ý việc làm. Dịch vụ kết nối trực tiếp với PostgreSQL để đọc dữ liệu CV và công việc, tính toán vector nhúng sử dụng mô hình \textit{VoVanPhuc/sup-SimCSE-VietNamese-phobert-base}, áp dụng thuật toán lọc phân tầng, và lưu kết quả gợi ý vào database. Việc tách biệt Recommend Service thành một dịch vụ độc lập mang lại hai lợi ích chính: cho phép sử dụng môi trường runtime phù hợp với các thư viện học máy Python, đồng thời có thể cập nhật hoặc thay thế mô hình AI mà không ảnh hưởng đến hoạt động của API Server.

Việc quản lý tệp tin được thực hiện thông qua Firebase Storage SDK. API Server xử lý logic nghiệp vụ liên quan đến tệp tin như kiểm tra định dạng và kích thước, trong khi việc lưu trữ vật lý được ủy quyền cho dịch vụ đám mây của Firebase. Giải pháp này giúp giảm tải cho máy chủ backend và tận dụng hạ tầng phân phối nội dung toàn cầu của Firebase.
